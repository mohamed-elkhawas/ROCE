\chapter{Architecture}


\section{Address Mapper}
\hspace{10mm}A Input:  the full req.\\ dfdf

%\begin{figure}[ht]
 %   \centering
  %  \includegraphics[width=0.7\textwidth]{Images/Buck Converter Circuit.png}
   % \caption{Buck Converter}
    %\label{fig1}
%\end{figure}

%\hspace{10mm}For mathematical expressions of Inductance and Capacitance refer eq. \ref{eq1} and eq. \ref{eq2}


desicion:
The main diff is that thread fair algo has higher priority to read requests while RLDP always 
focus on row hits,thus, it has too much transitions between reads and writes, so it leaves out 
the read queue and stall the system 


1- The algo continues to write always if there is a row hit
2- if there is a row hit, then we switch to read queue.
3- Its the problem that we leave read queue if there is a row hit on write queue, so we
decided to insert a new condition to delay this switching in future cycles and priotirzes the read queue higher than write drain.


we can switch to thread-fair algo later. we can jsut edit our flowchart in the lower branches 
and turn it into thread-fair!






\section{The memory controller structure:}
%\begin{itemize}
 %   \item Front-end structure
  %  \begin{itemize}
   %     \item Transaction controller
    %        \begin{itemize}
     %           \item Mapper       
      %          \item Returner
       %         \item Over flow stopper
        %        \item Request saver
         %   \end{itemize} 
        
        
        %\item Modified FIFO (per bank)
        %\item Bank scheduler (per bank)
        

    %\end{itemize}        
    %\item Back-end structure
     %   \begin{itemize}
      %      \item Arbiter       
       %     \item Timing controller
        %    \item Burst handler
        %\end{itemize}
%\end{itemize}






\subsection{Mapper block:}
 Applies mapping scheme and gives every request special index
\subsection{Returner block:} 
 Waits for specific request index to return then it send it out of the controller if other requests come it saves it in it until its turn to be sent out comes
\subsection{Over flow stopper block: }  
 Make shore that there are no two requests in the controller have the same index
\subsection{Request saver block:} 
 Stores the request that came out from the mapper when the Modified FIFO block is full then send it as soon as there is place at the FIFO
\subsection{Modified FIFO block:}
 It has two FIFO buffers inside one for the write request data (has low entries number) and another for the request type,
 address and index (with high entries number). It saves and assign the data if it’s write request only
































\section{Refresh Management}
\hspace{10mm}A Buck converter takes the voltage from a DC source and converts the voltage of supply into lower DC voltage level. Some devices need a certain amount of voltage to run the device. Too much of power can destroy the device or less power may not be able to run the device.This output voltage is achieved by chopping the input voltage with a series of connected switches that apply pulses to an averaging inductor and capacitor circuit.

%\begin{figure}[ht]
 %   \centering
  %  \includegraphics[width=0.7\textwidth]{Images/Buck Converter Circuit.png}
   % \caption{Buck Converter}
    %\label{fig1}
%\end{figure}

%\hspace{10mm}For mathematical expressions of Inductance and Capacitance refer eq. \ref{eq1} and eq. \ref{eq2}