%% LyX 2.3.6.1 created this file.  For more info, see http://www.lyx.org/.
%% Do not edit unless you really know what you are doing.
\documentclass[english]{report}
\usepackage[T1]{fontenc}
\usepackage[latin9]{inputenc}
\usepackage[a4paper]{geometry}
\geometry{verbose,tmargin=1in,bmargin=1in,lmargin=1in,rmargin=1in}
\setcounter{secnumdepth}{3}
\setcounter{tocdepth}{3}
\usepackage{babel}
\usepackage{booktabs}
\usepackage{graphicx}
\usepackage{nomencl}
\usepackage{comment}
% the following is useful when we have the old nomencl.sty package
\providecommand{\printnomenclature}{\printglossary}
\providecommand{\makenomenclature}{\makeglossary}
\makenomenclature
\usepackage[unicode=true,pdfusetitle,
 bookmarks=true,bookmarksnumbered=false,bookmarksopen=false,
 breaklinks=false,pdfborder={0 0 0},pdfborderstyle={},backref=false,colorlinks=false]
 {hyperref}

\makeatletter

%%%%%%%%%%%%%%%%%%%%%%%%%%%%%% LyX specific LaTeX commands.
%% Because html converters don't know tabularnewline
\providecommand{\tabularnewline}{\\}

%%%%%%%%%%%%%%%%%%%%%%%%%%%%%% User specified LaTeX commands.
\pagenumbering{roman}
\usepackage{cite}
\usepackage[nottoc,notlof,notlot]{tocbibind}
\AtBeginDocument{\def\nomname{List of Abbreviations and Symbols}}
\renewcommand{\nomgroup}[1]{%
\ifthenelse{\equal{#1}{A}}{\item[\textbf{Abbreviations}]}{%
\ifthenelse{\equal{#1}{B}}{\item[\textbf{Symbols}]}{%
\ifthenelse{\equal{#1}{C}}{\item[\textbf{Subscripts}]}
{}
}% matches Subscripts
}% matches Symbols
}% matches Abbreviations

\makeatother

\def\eqdeclaration#1{, see equation\nobreakspace(#1)}
\def\pagedeclaration#1{, page\nobreakspace#1}
\def\nomname{Nomenclature}

\begin{document}
\title{RDMA Over Converged Ethernet}
\author{Mohammed Hussien, , , et al.\\
Computer and Systems Engineering Department\\
Faculty of Engineering, Ain Shams University\\
Cairo, Egypt}

\maketitle
\newpage{}
\begin{center}
\includegraphics[scale=0.1]{\string"Images/ASU FOE LOGO\string".png}
\par\end{center}

\begin{center}
Ain Shams University, Faculty of Engineering
\par\end{center}

\begin{center}
Electronics and Communications Engineering Department
\par\end{center}

\begin{center}
Cairo, Egypt
\par\end{center}

\vspace{5cm}

\begin{center}
\textbf{\Large{}RDMA Over Converged Ethernet}{\Large\par}
\par\end{center}

\begin{center}
{\large{}A Report Submitted in Partial Fulfillment of the Requirements
of the Degree of Bachelor of Science in Electronics and Communications
Engineering }\textbf{\large{}\vspace{3cm}
}{\large\par}
\par\end{center}

\begin{center}
\begin{tabular}{cccc}
\toprule 
Mohammed Hussien & 1601160 & Member 2 & Code 2\tabularnewline
\midrule
Member 3 & Code 3 & Member 4 & Code 4\tabularnewline
\midrule 
Member 5 & Code 5 & Member 6 & Code 6\tabularnewline
\midrule 
Member 7 & Code 7 & Member 8 & Code 8\tabularnewline
\midrule 
Member 9 & Code 9 & Member 10 & Code 10\tabularnewline
\bottomrule
\end{tabular}
\par\end{center}

\begin{center}
\vspace{2cm}
\par\end{center}

\begin{center}
Supervised By
\par\end{center}

\begin{center}
Prof. Dr. Ashraf Salem
\par\end{center}

\begin{center}
\medskip{}
\par\end{center}

\begin{center}
Cairo 2021
\par\end{center}

\thispagestyle{empty}

\newpage{}

\chapter*{Declaration}

We hereby certify that this project submitted as part of our partial
fulfillment of BSc in Electronics and Communications Engineering is
entirely our own work, that we have exercised reasonable care to ensure
its originality, and does not to the best of our knowledge breach
any copyrighted materials, and have not been taken from the work of
others and to the extent that such work has been cited and acknowledged
within the text of our work.

Signed
\begin{center}
\begin{tabular}{ccc}
\toprule 
Name & University ID & Signature\tabularnewline
\midrule
Mohammed Hussien & 1601160 & \tabularnewline
\midrule 
Member 2 & Code & \tabularnewline
\midrule 
Member 3 & Code & \tabularnewline
\midrule 
Member 4 & Code & \tabularnewline
\midrule 
Member 5 & Code & \tabularnewline
\midrule 
Member 6 & Code & \tabularnewline
\midrule 
Member 7 & Code & \tabularnewline
\midrule 
Member 8 & Code & \tabularnewline
\midrule 
Member 9 & Code & \tabularnewline
\midrule 
Member 10 & Code & \tabularnewline
\bottomrule
\end{tabular}
\par\end{center}

Date: Day of 24\textsuperscript{th} of July, in the year 2021.

\thispagestyle{empty}

\newpage{}

\chapter*{Acknowledgment}

Thank and acknowledge your advisor, family and friends.

\thispagestyle{empty}\newpage{}
\begin{abstract}
Write here a brief summary of your thesis.
\end{abstract}
\newpage{}

\tableofcontents{}

\newpage{}

\listoffigures
\newpage{}

\listoftables

\newpage{}

\settowidth{\nomlabelwidth}{$\alpha_{s}$}
\printnomenclature{}

\newpage{}

\nomenclature[A]{AI}{Artificial Intelligence}\nomenclature[A]{DQN}{Deep Q-Network}\nomenclature[A]{ReLU}{Rectified Linear Unit}
\nomenclature[A]{SPI}{Serial Peripheral Interface}\nomenclature[A]{SARSA}{State\textendash action\textendash reward\textendash state\textendash action}\nomenclature[A]{MSE}{Mean Square Error}\nomenclature[A]{MAE}{Mean Absolute Error}\nomenclature[A]{MBE}{Mean Bias Error}\nomenclature[A]{CNN}{Convolutional Neural Network}\nomenclature[A]{SGD}{Stochastic Gradient Descent}
\nomenclature[B]{$\chi^{2}$}{Chi-Squared} \nomenclature[B]{$\alpha_{s}$}{Specific attenuation}
\nomenclature[C]{r}{Received}

\pagenumbering{arabic}
\setcounter{page}{1}

\chapter{Introduction}

\chapter{Project Description}
\section{Detailed project description}
%Begin from here


\subsection{Image:}

\subsubsection{Histogram:}


\subsubsection{Color Layout:}




\subsubsection{Mean Color:}



\subsection{Video: }
\subsubsection{Histogram:}




\section{Beneficiaries of the project}
%Begin from here





\section{Detailed analysis}
%Begin from here















\section{Techniques description}
\subsection{Images techniques}

\subsubsection{Mean color}


\subsubsection{Histogram}

\


%\begin{figure}[H]
 %   \centering
  %  \includegraphics[width=40mm,height=20mm]{Images/eq.png}
  %  \caption{histogram distance equation}
  %\end{figure}

\subsubsection{Color Layout}

\subsection{Videos techniques}
%Begin from here
\subsubsection{Feature extraction technique}

\subsubsection{Distance calculation}

\chapter{Literature Review (Sub Hierarchies)}

\chapter{System Design And Architecture}
\section{System architecture}
%Begin from here





\section{Multimedia database design}
%Begin from here
\subsection{The purpose of our data base}
\subsection{Required information}

\subsection{Database Schema}
%\begin{itemize}
 %   \item The first tables is "Image" which holds the information for each image.        
  %  \item The second is "Video" which holds only the ID and Path of each video.
   % \item The third table is "Key Frame".
  %  this table holds all the key frames of all stored videos and relate each key fram to its video using the video ID.
%\end{itemize}
\vskip 0.2in

\subsection{Primary keys and foriegn keys}


\section{System design}
%Begin from here




\section{Testing scenarios and results}
%Begin from here

\chapter{End User Guide}

\section{Steps To use the project}

\section{Features}


%And of course our project itself is open source with a [https://github.com/mohammedBadawi/Multimedia-Project-Document.git][dill] on GitHub.

% old chapters are Introduction BlockDiagram Architecture.
%references Zhao Zhang, Zhichun Zhu and Xiaodong Zhang, "A permutation-based page interleaving scheme to reduce row-buffer conflicts and exploit data locality," Proceedings 33rd Annual IEEE/ACM International Symposium on Microarchitecture. MICRO-33 2000, Monterey, CA, USA, 2000, pp. 32-41, doi: 10.1109/MICRO.2000.898056.
%\bibitem{Qx2}
%Young-Suk Moon, Yongkee Kwon, Hong-Sik Kim, Dong-gun Kim, Hyungdong Hayden Lee and Kunwoo Park,"The Compact Memory Scheduling Maximizing Row Buffer Locality,"
%\end{thebibliography}







\renewcommand\bibname{References}
\begin{thebibliography}{1}
\bibitem{key-1}Better use BibTeX and delete this section.
\end{thebibliography}

\appendix

\chapter{First Appendix}

\chapter{Second Appendix}
\end{document}





\begin{comment}
\section{Purpose of the controller}

\hspace{10mm}A passive component designed to resist changes in current. Inductors are often referred to as “AC resistors”. The ability
to resist changes in current and store energy in its magnetic field account for the bulk of the useful properties of inductors. Current passing through an inductor will produce a magnetic field. A changing magnetic field induces a voltage which opposes the field-producing current. This property of impeding changes of current is known as inductance. The voltage induced across an inductor by a change of current is defined as:
$$V = L \frac{di}{dt}$$



\section{Features}


%\hspace{10mm}The DDR3 controller supports the following features: \begin{abstract}
 %   Supports JEDEC standard JESD79-3C – DDR3 compliant devices
%\end{abstract}


\begin{itemize}
    \item 33-bit address for 8 GB of address space.
    \item 16/32/64-bit data bus width support.
    \item CAS latencies: 5, 6, 7, 8, 9, 10, and 11.
    \item 1, 2, 4, and 8 internal banks.
    \item Burst Length: 8.
    \item Burst Type: sequential.
    \item 8GB address space available over one or two chip selects.
    \item Page sizes: 256, 512, 1024, and 2048-word.
    \item SDRAM auto initialization from reset or configuration change.
    \item Self-refresh mode.
    \item Prioritized refresh scheduling.
    \item Programmable SDRAM refresh rate and backlog counter.
    \item Programmable SDRAM timing parameters.
    \item Big and little endian modes.
    \item ECC on SDRAM data bus.
    \item 8-bit ECC per 64-bit data quanta without additional cycle latency.
    \item Two latency classes supported.
    \item UDIMM Address mirroring is not supported.
\end{itemize}
 

\section{The DDR5 technology}
\begin{enumerate}
    \item Every command between the controller and the request must be in bursts (16 (columns or normal requests)).
    \item There is small timing constrains between the commands in different bank groups.
    \item The timing constrains between the two commands in the same bank is the largest then the two banks in the same bank group then the different bank groups.
    \item There is timing constrains between writing to the memory and reading from it.
\end{enumerate}





















\chapter{Architecture}


\section{Address Mapper}
\hspace{10mm}A Input:  the full req.\\ dfdf

%\begin{figure}[ht]
 %   \centering
  %  \includegraphics[width=0.7\textwidth]{Images/Buck Converter Circuit.png}
   % \caption{Buck Converter}
    %\label{fig1}
%\end{figure}

%\hspace{10mm}For mathematical expressions of Inductance and Capacitance refer eq. \ref{eq1} and eq. \ref{eq2}


desicion:
The main diff is that thread fair algo has higher priority to read requests while RLDP always 
focus on row hits,thus, it has too much transitions between reads and writes, so it leaves out 
the read queue and stall the system 


1- The algo continues to write always if there is a row hit
2- if there is a row hit, then we switch to read queue.
3- Its the problem that we leave read queue if there is a row hit on write queue, so we
decided to insert a new condition to delay this switching in future cycles and priotirzes the read queue higher than write drain.


we can switch to thread-fair algo later. we can jsut edit our flowchart in the lower branches 
and turn it into thread-fair!






\section{The memory controller structure:}
%\begin{itemize}
 %   \item Front-end structure
  %  \begin{itemize}
   %     \item Transaction controller
    %        \begin{itemize}
     %           \item Mapper       
      %          \item Returner
       %         \item Over flow stopper
        %        \item Request saver
         %   \end{itemize} 
        
        
        %\item Modified FIFO (per bank)
        %\item Bank scheduler (per bank)
        

    %\end{itemize}        
    %\item Back-end structure
     %   \begin{itemize}
      %      \item Arbiter       
       %     \item Timing controller
        %    \item Burst handler
        %\end{itemize}
%\end{itemize}






\subsection{Mapper block:}
 Applies mapping scheme and gives every request special index
\subsection{Returner block:} 
 Waits for specific request index to return then it send it out of the controller if other requests come it saves it in it until its turn to be sent out comes
\subsection{Over flow stopper block: }  
 Make shore that there are no two requests in the controller have the same index
\subsection{Request saver block:} 
 Stores the request that came out from the mapper when the Modified FIFO block is full then send it as soon as there is place at the FIFO
\subsection{Modified FIFO block:}
 It has two FIFO buffers inside one for the write request data (has low entries number) and another for the request type,
 address and index (with high entries number). It saves and assign the data if it’s write request only
































\section{Refresh Management}
\hspace{10mm}A Buck converter takes the voltage from a DC source and converts the voltage of supply into lower DC voltage level. Some devices need a certain amount of voltage to run the device. Too much of power can destroy the device or less power may not be able to run the device.This output voltage is achieved by chopping the input voltage with a series of connected switches that apply pulses to an averaging inductor and capacitor circuit.

%\begin{figure}[ht]
 %   \centering
  %  \includegraphics[width=0.7\textwidth]{Images/Buck Converter Circuit.png}
   % \caption{Buck Converter}
    %\label{fig1}
%\end{figure}

%\hspace{10mm}For mathematical expressions of Inductance and Capacitance refer eq. \ref{eq1} and eq. \ref{eq2}

\end{comment}